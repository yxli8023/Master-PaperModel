\linespread{1.6}%行间距设置
\numberwithin{equation}{section}%公式按照章节标号
\renewcommand\thefigure{\thesection.\arabic{figure}}%图片按章节标号
\makeatletter
\@addtoreset{figure}{section}
\renewcommand{\figurename}{图}               % 对图表中的Fig进行中文翻译变为图
\renewcommand{\contentsname}{目$\qquad$录}           % 对Contents进行汉化为目录
\renewcommand\listfigurename{插\ 图\ 目\ 录} % 对List of Figures进行汉化为插图目录
\renewcommand\listtablename{表\ 格\ 目\ 录}  % 对List of Tables进行汉化表格目录
\setcounter{tocdepth}{2}%目录章节深度设置

%\setlength{\parskip}{0.5\baselineskip}% 设置空行换行后,上下两段文字间距
%\titlespacing*{section}{0pt}{9pt}{0pt}% 设置标题与段落间距
%============================= 目录设置 ======================
\titlecontents{chapter}[1.5em]{\zihao{3}\bf }{\contentslabel{1.5em}}{\hspace*{-2em}}{\titlerule*[5pt]{$\cdot$}\contentspage}
\titlecontents{section}[3.3em]{\zihao{-3}\bf }{\contentslabel{1.8em}}{\hspace*{-2.3em}}{\titlerule*[5pt]{$\cdot$}\contentspage}
\titlecontents{subsection}[2.5em]{\zihao{3}}{\thecontentslabel{$\quad$}}{}{\titlerule*[5pt]{$\cdot$}\contentspage}

%==========================================================================
\renewcommand{\baselinestretch}{1.5} %行间距1.5倍
%===============================页眉页脚设置================================
\fancypagestyle{plain}{
%\fancyhead[RE]{\leftmark} % 在偶数页的右侧显示章名
\fancyhf{}
%\fancyhead[C]{\rightmark} %页眉居中显示章节名
\fancyhead[C]{高温超导异质结中的旋转对称破缺和部分马约拉纳角态} %页眉居中设置论文标题
\fancyfoot[C]{第\thepage 页}
%\fancyhead[LO]{\rightmark} % 在奇数页的左侧显示小节名
%\fancyhead[LE,RO]{~\thepage~} % 在偶数页的左侧,奇数页的右侧显示页码
% 设置页脚:在每页的右下脚以斜体显示书名
%\fancyfoot[RO,RE]{\it \text{右下角内容}} %右下角增加\text{}中的内容
\renewcommand{\headrulewidth}{1.5pt} % 页眉与正文之间的水平线粗细
\renewcommand{\footrulewidth}{1.5pt}
%\renewcommand{\footrulewidth}{0pt}
%\renewcommand\headrule{\hrule width \hsize height 2pt \kern 2pt \hrule width \hsize height 0.4pt}%页眉上双线
}
%=============== 全局字体设置 ===============
\renewcommand{\songti}{\CJKfontspec{STSong}}% 华文宋体

%===============================  引用上标设置 ===========
\makeatletter
\def\@cite#1#2{\textsuperscript{[{#1\if@tempswa , #2\fi}]}}
\makeatother
%============= 设置自己的简短命令 ============================

