%\setcounter{page}{1} \renewcommand{\thepage}{\wuhao\Roman{page}} % 页码设置
{\centering{\zihao{3}{\bf 高温超导异质结中的旋转对称破缺和部分马约拉纳角态}}\\}
\bigskip
{\zihao{-4}
	\begin{center}
		\begin{tabular}{l}
			专业名称:$\quad$凝聚态物理$\quad\qquad$$\quad\qquad$$\quad$$\quad$$\quad$$\quad$$\quad$$\quad$\\
			申请者: $\quad\quad$李玉轩$\quad\qquad$$\quad\qquad$$\quad\qquad$\\
			导师姓名:$\quad$周涛\quad 教授$\quad\qquad$$\quad\qquad$$\quad\qquad$\\
		\end{tabular}
\end{center}}
\bigskip
\bigskip
\bigskip
{\flushleft{\zihao{-3}\heiti 摘\quad 要}}
\addcontentsline{toc}{section}{摘\quad 要}

%\setlength{\baselineskip}{26pt}
%\thispagestyle{empty}
 近些年来,拓扑绝缘体在理论与实验上都有了非常迅速的发展。最近研究人员提出了一类新的高阶拓扑系统,相比于传统拓扑系统,一个$d$维的$n$阶拓扑系统,会在$(d-n)$维的边界上出现拓扑保护的边界态,而非在$(d-1)$维边界上。比如,在二维二阶拓扑超导体中,马约拉纳零能态不会出现在一维的边界上,而是在零维的角落中。
 
 之前从唯象模型出发,研究人员提出可以在二维拓扑绝缘体与高温超导体异质结结构中实现高阶拓扑超导体。本论文中,我们从微观模型出发,在考虑了超导体与二维拓扑绝缘体的同时,并加入了两者间的耦合,在理论上重新研究了该系统。
 
 首先,通过对谱函数的计算,在考虑了耦合之后与唯象模型相比能谱发生了较大的变化,马约拉纳零能态并不会出现在系统的所有角落。对半无限大系统的能带计算表明,在$i_y=N_y$边界上的边界态始终未打开能隙,而其余三个边界上都存在超导诱导出的能隙,这与唯象模型的结果是完全不同的。
 
 接下来,我们研究了拓扑绝缘体中动量空间的超导序参量,此时序参量的$\mathcal{C}_4$对称性被破坏,因为其中不仅存在与超导层配对形式相同的$d$波分量,还存在$p$波分量。对实空间序参量的计算发现相比于$i_x=1$,$i_x=N_x$,$i_y=1$,在$i_y=N_y$这条边界上$d$波序参量非常小,这与半无限系统的能带计算结论自洽。
 
 最后利用格林函数运动方程方法求解了动量空间中的反常格林函数,它可以定性的反应超导序参量的特征,我们发现可以将其表示为奇函数与偶函数部分,这与之前动量空间序参量计算得到同时包含$d$波与$p$波分量的结论一致。
 
 通过数值以及解析计算,我们成功的揭示了在异质结系统实现高阶拓扑超导体的方案中,马约拉纳零能态并非出现在每个角落中,这对之后在实验上实现以及观测高阶拓扑超导体中的马约拉纳零能态具有重要意义。\\
\quad\noindent{\zihao{-4}\heiti 关键词:}拓扑绝缘体;高阶拓扑超导;马约拉零能态;

\newpage
{\centering \zihao{-3}{\heiti Rotational symmetry breaking and partial Majorana corner states in a heterostructure based on high-T$_c$ superconductors}}
\bigskip
{\zihao{-3}
	\begin{center}
		\begin{tabular}{l}
			Major:$\quad\quad$Condensed Matter Physics$\quad\qquad$$\quad\qquad$\\
			Name:$\quad\quad$YuXuan-Li$\quad\qquad$$\quad\qquad$\\
			Supervisor: Tao Zhou$\quad\qquad$\\
		\end{tabular}
\end{center}}
\bigskip
\bigskip
\bigskip

{\flushleft{\zihao{-3}\bf ABSTRACT}}

\addcontentsline{toc}{section}{ABSTRACT}
%\linespread{1.4}\zihao{-4}
%\thispagestyle{empty}%将本页的格式完全清除
 In recent years, topological insulators have developed rapidly in theory and experiment.  A new class of system,
 dubbed as higher-order topological system, has been proposed. In contrast to conventional topological system, $n$th-order$(n\geq 2)$ topological system in $d$ dimensions host $(d-n)$-dimensional edge states, rather than $(d − 1)$-dimensional edges. For example, in 2D second-order topological superconductors, the edge modes manifest themselves as 0D Majorana excitations localized at the corners, instead of 1D edges.
 
Previously, starting from the phenomenological model, the researchers proposed that higher-order topological superconductors can be achieved in a heterojunction system which combine with two-dimensional topological insulator and high-temperature superconductor. In this paper we theoretically re-study this system from a microscopic model. Except consider the superconductor and the two-dimensional topological insulator, the coupling between them is included.

First of all, the calculation of spectral function indicates that after considering the coupling, the results have changed significantly compare with the phenomenological model. And the Majorana zero modes does not present at every corners of the system. The calculation of the energy band for  cylinder geometry system show that the edge state on the $i_y=N_y$ is gapless, the remaining three edges are gapped. The results are inconsistent with phenomenological model.

Next, we study the $\mathcal{C}_4$ symmetry of order parameter within the momentum space for topological insulator,  which is broken due to the coexistence of $d$ and $p$ wave components. The calculation of the real space order parameter found that compared to $i_x=1$, $i_x=N_x$, $i_y=1$, the $d$ wave order parameter is extremely small on the edge of $i_y=N_y$. This result is consistent with cylinder energy band.

Finally, by using the Green's function equation of motion, the anomalous Green's function which can qualitatively reflect the characteristics of the superconducting order parameter is solved in momentum space. It can be divided into even and odd part, which is consistent with numerical calculation of order parameter. 

Combining numerical and analytical calculations, we have successfully revealed that the Majorana zero state does not appear in every corner for  heterojunction system. These results are significance for the subsequent experiment in higher-order topological superconductors.
 \\
\noindent\textbf{\zihao{-4} Keywords:} Topological insulator, Higher-order topological superconductor, Majorana zero mode