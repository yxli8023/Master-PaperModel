%\setcounter{page}{1} \renewcommand{\thepage}{\wuhao\Roman{page}} % 页码设置
{\centering{\zihao{3}{\bf 高温超导异质结中的旋转对称破缺和部分马约拉纳角态}}\\}
\bigskip
{\zihao{-4}
	\begin{center}
		\begin{tabular}{l}
			专业名称:$\quad$凝聚态物理$\quad\qquad$$\quad\qquad$$\quad$$\quad$$\quad$$\quad$$\quad$$\quad$\\
			申请者: $\quad\quad$李玉轩$\quad\qquad$$\quad\qquad$$\quad\qquad$\\
			导师姓名:$\quad$周涛\quad 教授$\quad\qquad$$\quad\qquad$$\quad\qquad$\\
		\end{tabular}
\end{center}}
\bigskip
\bigskip
\bigskip
{\flushleft{\zihao{-3}\heiti 摘\quad 要}}
\addcontentsline{toc}{section}{摘\quad 要}

%\setlength{\baselineskip}{26pt}
%\thispagestyle{empty}
 近些年来,拓扑绝缘体在理论与实验上都有了非常迅速的发展,虽然其体态能带存在能隙,但是边界上存在无能隙的边界态,电子可以在边界上无耗散的运动,这就是拓扑绝缘体的体边对应关系。对一个超导系统,描述其准粒子的BdG哈密顿量与描述拓扑绝缘体能带的哈密顿量之间有着相似之处,超导体的能隙即对应着拓扑绝缘体的带隙,因此拓扑超导体同样引起了研究人员的浓厚的兴趣。对于拓扑体系,体边对应关系是其最主要的特征之一,拓扑超导体的体态虽然存在超导能隙,但边界上会出现受拓扑保护的边界态,由于BdG哈密顿量满足粒子空穴对称性,这个边界态与马约拉纳费米子密切相关。马约拉纳费米子的反粒子是它本身,通常满足非阿贝尔统计,这使得其在拓扑量子计算方面具有较好的应用潜力。最近研究人员提出了一种新的对应关系,体角对应:对一个二维的体系,其拓扑保护的边界态出现在比系统低两维的角落中,研究人员将这种新拓扑物态称为高阶拓扑。一个二维的拓扑超导系统,当边界态出现在系统的角落处时,被称为高阶拓扑超导体。
 
 在之前的研究中,基于唯象模型,研究人员提出可以在二维拓扑绝缘体和高温超导体的异质结系统中实现高阶拓扑超导体。理论上通过将超导电子配对项直接加入到拓扑绝缘体哈密顿量中,研究表明在一个正方晶格点阵中,马约拉纳束缚态会出现在系统的所有角落中。本论文中,我们从微观模型出发,对于一个异质结结构,同时考虑了超导体与二维拓扑绝缘体,并考虑了两者之间的耦合,在理论上重新研究了该系统。首先,通过对谱函数的计算,在考虑了耦合之后我们发现低能时谱函数主要来自于超导体的贡献,当能量较高时由于拓扑绝缘体与超导体之间的耦合,与唯象模型相比能谱发生了较大的变化。通过计算实空间中零能量的局域电子态密度,我们发现此时并非在正方晶格点阵的每一个角落都会出现马约拉纳零能态。对半无限大系统的能带计算表明,在$i_y=N_y$边界上的边界态始终未打开能隙,而其余三个边界上都存在超导诱导出的能隙,这与唯象模型的结果是完全不同的。正是由于上边界的边界态未打开能隙,从而不会在其相邻的边界的角落处形成束缚态。
 
 接下来,对于拓扑绝缘体,我们计算了动量空间的超导序参量,我们发现序参量的$C_4$对称性被破坏,这是因为其中不仅存在与超导层配对形式相同的$d$波分量,还存在$p$波分量。对实空间序参量的计算发现相比于$i_x=1$,$i_x=N_x$,$i_y=1$,在$i_y=N_y$这条边界上$d$波序参量非常小,这与半无限系统的能带计算结果自洽。最后我们利用格林函数运动方程方法求解了动量空间中的反常格林函数,它可以定性的反应超导序参量的特征,我们发现其可以表示为奇函数与偶函数部分,与动量空间序参量计算得到同时包含$d$波与$p$波分量的结论一致。
 
 通过数值以及解析求解计算,我们成功的揭示了在通过异质结系统利用超导的近邻效应来实现高阶拓扑超导体时,对于一个正方晶格点阵,马约拉纳角态并不是在所有的角落中都会出现,这对之后在实验上实现以及观测高阶拓扑超导体中的马约拉纳角态和实现拓扑量子计算都具有重要的意义。\\
\quad\noindent{\zihao{-4}\heiti 关键词:}高阶拓扑超导;马约拉纳角态;拓扑绝缘体;

\newpage
{\centering \zihao{-3}{\heiti Rotational symmetry breaking and partial Majorana corner states in a heterostructure based on high-T$_c$ superconductors}}
\bigskip
{\zihao{-3}
	\begin{center}
		\begin{tabular}{l}
			Major:$\quad\quad$Condensed Matter Physics$\quad\qquad$$\quad\qquad$\\
			Name:$\quad\quad$YuXuan-Li$\quad\qquad$$\quad\qquad$\\
			Supervisor: Tao Zhou$\quad\qquad$\\
		\end{tabular}
\end{center}}
\bigskip
\bigskip
\bigskip

{\flushleft{\zihao{-3}\bf ABSTRACT}}

\addcontentsline{toc}{section}{ABSTRACT}
%\linespread{1.4}\zihao{-4}
%\thispagestyle{empty}%将本页的格式完全清除
 In recent years, topological insulators have developed very rapidly in theory and experiment. Due to the similarity between the paired energy gap of superconductors and the energy gap of insulators, topological superconductors have also attracted the attention of researchers. For a topological system, the bulk-boundary correspondence is one of its most important features.  Although the bulk of a topological superconductor has an electron pairing energy gap, there are topologically protected edge states on the boundary, which is the symmetry of the particles and holes in the superconductor. Existence, this boundary state is closely related to Majorana fermion, Majorana fermion itself is also its antiparticle, and it satisfies non-Abelian statistics and has huge potential in topological quantum computing. Recently, researchers have discovered a new correspondence between bulk corner, that is, for a two-dimensional system, the edge state  appears in the corner of two dimensions which lower than the system, in a two-dimensional superconducting system, there will be Majorana corner states. 
 
 Previously, based on the phenomenological model, researchers proposed that high-order topological superconductors can be realized in a heterojunction system of two-dimensional topological insulators and high superconductors. By adding superconducting electron pairs directly to the topological insulator Hamiltonian, the study The results show that for a tetragonal sample, Majorana bound states will appear in every corner of the system. In this paper, we start from the microscopic model, we consider the superconductor and the two-dimensional topological insulator at the same time, and take the coupling between the two into account, and re-study the system theoretically. First, we calculated the spectral function and found that after considering the coupling, the spectral function at low energies mainly comes from the contribution of the superconductor. When the energy is higher, due to the combination between the topological insulator and the superconductor, the energy spectrum has occurred compared with the phenomenological model. Big change. Next, from the open-boundary energy band diagram and the results of the local and electronic density of states in real space, we find that unlike the phenomenological model, the Majorana corner state does not appear in every corner of the system, because the $i_y=N_y$ in the system, the boundary state on the edge does not open the energy gap, and there are energy gaps induced by superconductivity on the other three boundaries. There is no mass with the opposite sign between this boundary and the adjacent boundary, so that corner states cannot be formed at the corners between adjacent sides.  From the calculation of the order parameters in momentum space and real space, we know that the electron pairing induced by the proximity effect in the topological insulator can include both the $d$ and  $p$ wave, and their interaction will make the electron pairing on a certain boundary becomes very small, so that the boundary state of the topological insulator cannot open the energy gap. The result of the real space order parameter also confirms this point. Finally, we use the Green's function equation of motion method to solve the anomalous Green's function in the momentum space. We find that it can be expressed as odd function and even function part, which corresponds to the $p$ wave and $d$ wave electron paired in numerical calculations. The analytical and numerical results are identical. 
 
 Through numerical and theoretical analysis, we have successfully revealed that the Majorana corner state does not appear in all corners when the proximity effect of superconductors is used to realize high-order topological superconductors through heterojunction systems. It is of great significance to realize experimentally and observe Majorana corner states in high-order topological superconductors and to use them to realize topological quantum calculations. \\
\noindent\textbf{\zihao{-4} Keywords:} higher-order topological superconductivity,Majorana,topological insulator,
quasiparticle