\section{总结与展望}
高阶拓扑超导体与传统的一阶拓扑超导体具有不同的性质,其马约拉纳束缚态会出现在比系统维度低两维甚至三维的边界上。在论文的第一章我们主要介绍了拓扑绝缘体,超导体,以及高阶拓扑超导体的基本概念,并从低能边界态理论出发介绍了在2D系统的拐角处,高阶拓扑超导体产生的机制。科研人员利用2D拓扑绝缘体与各向异性配对的$d$-波超导体构成的异质结来实现高阶拓扑超导体,从而在系统的角落中存在马约拉纳拐角态。同时研究人员也研究了利用常规的$s$-波超导体与一个面内的Zeeman场,当与2D拓扑绝缘体形成异质结系统的时候,也可以在拐角处形成马约拉纳拐角态,但此时面内的Zeeman场破坏了时间反演对称性,所以在每个拐角处只存在一个马约拉纳拐角态。随后我们介绍了在通过异质结结构研究一阶拓扑超导体的时候,层间的耦合会对诱导出的电子配对对称性存在一定影响,研究人员发现除了超导本来的配对形式,因为晶体结构之间的不匹配,还会存在其他形式的电子配对会在近邻层产生,这种情况下形成的其实是混合类型的超导配对。

在论文的第二章,基于异质结结构的复杂性,在唯象模型研究的基础上,我们给出了一个研究高阶拓扑超导体的微观模型,主要考虑了超导体与2D拓扑绝缘体之间的耦合作用,并介绍了具体的研究方法。第三章我们主要研究了微观模型的能带以及边界态谱函数,发现体系的$\mathcal{C}_4$对称性破缺,而且特定位置的边界态并没有因为近邻效应的存在而打开能隙,并通过计算零能态的局域电子密度,发现在实空间中,未打开能隙的边界上确实在其拐角处不存在马约拉纳拐角态。在第四章中我们计算了动量空间中以及实空间中的序参量,进一步确定了通过近邻效应诱导的电子配对的对称性,发现通过近邻效应诱导的电子配对同时包含了单重态与三重态两种成分。对实空间$d$-波电子配对的计算也发现了在体系的上边界,通过近邻效应诱导出的电子配对是很小的,所以这再次验证了我们之前关于能带以及谱函数的计算。在第五章中,我们利用运动方程的方法,求解了单个轨道的反常格林函数,它可以定性的反应系统中电子的配对,我们发现此时可以将电子的配对分成奇宇称通道和偶宇称通道,通过对费米面以下的积分,计算得到的电子配对结果与我们利用数值对角化矩阵计算超导序参量的结果是完全相同的。我们同时也利用准粒子的边界哈密顿量计算了系统边界上的反常格林函数,发现在$i_y=N_y$这个边界上,电子配对的贡献是非常小的。

综上所述,我们从数值以及解析两种不同的方式,利用微观模型研究了2D拓扑绝缘体与$d$-波超导体形成的异质结中,结果表明马约拉纳拐角态只会出现在部分角落中,而对于唯象模型研究则显示系统的所有角落中都会出现马约拉纳拐角态。我们的计算研究发现,这种不一致主要来源于近邻效应在2D拓扑绝缘体中诱导出来的是$(d+p)$-波的电子配对,正是因为这种配对破坏了$\mathcal{C}_4$对称性,而且不同边界上电子配对大小也是不同的,所以只会在部分角落中才会出现马约拉纳拐角态,这对之后实验上利用异质结结构探索高阶拓扑中的马约拉纳拐角态具有一定的价值,也为在高阶拓扑超导体中通过马约拉纳拐角态实现拓扑量子计算有一定的应用价值。



